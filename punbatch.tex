\documentclass{article}
\usepackage{color}
\usepackage[svgnames]{xcolor}
\usepackage[margin=0.6in]{geometry}
\usepackage{multirow,tabularx}
\newcolumntype{Y}{>{\centering\arraybackslash}X}
\renewcommand{\arraystretch}{2}
\begin{document}
\center {\color{black}\bf{\Huge{Punjab Technical University}}}


\center{\Large{{\huge{\bf{Scheme and Syllabus of}}}}}
\center{{\huge{\bf{ B.Tech}}}
\center{{\huge{ \bf{Computer Science Engineering}}}}
\center{{\Large{ \color{black}\bf{3} Semester effective for Batch }}}
\center{{\Huge{ \color{black}\emph{\linebreak\linebreak\linebreak\linebreak 3   SEMESTER}}}}
\linebreak\linebreak\linebreak
\begin{tabularx}{\linewidth}{|*{15}{X|}}
\hline
\multirow{2}{*}{Code} &\multicolumn{2}{|c|}{Course Details} &\multicolumn{3}{c|}{Load Allocation}&\multicolumn{2}{c|}{Mark Distribution} &\multicolumn{1}{|c|}{Total Marks}&\multicolumn{1}{|c|}{Credit} \\
\cline{4-8}
           & \multicolumn{2}{|c|}{} &  L & T & P & Internal & External & & \\

\hline
\cline{3-7}
           \multicolumn{1}{|c|}{BTCS 301} &\multicolumn{2}{c|}{Computer Architecture} &  3 & 2 & 1 & 40 & 60 & 100 & 2 \\
\hline

           \multicolumn{1}{|c|}{BTCS302} &\multicolumn{2}{c|}{Discrete Structures} &  3 & 2 & 1 & 40 & 60 & 100 & 2 \\
\hline

           \multicolumn{1}{|c|}{BTCS303} &\multicolumn{2}{c|}{Digital Circuits \& Logic Design} &  3 & 2 & 1 & 40 & 60 & 100 & 2 \\
\hline

           \multicolumn{1}{|c|}{BTCS 304} &\multicolumn{2}{c|}{Data Structures} &  3 & 2 & 1 & 40 & 60 & 100 & 2 \\
\hline

           \multicolumn{1}{|c|}{BTCS 305} &\multicolumn{2}{c|}{Object Oriented Programming Using C++} &  3 & 2 & 1 & 40 & 60 & 100 & 2 \\
\hline

           \multicolumn{1}{|c|}{BTCS308} &\multicolumn{2}{c|}{Digital Circuits \& Logic Design} &  0 & 0 & 1 & 25 & 25 & 50 & 2 \\
\hline

           \multicolumn{1}{|c|}{BTCS309} &\multicolumn{2}{c|}{Object Oriented Programming using C++} &  0 & 0 & 1 & 25 & 25 & 50 & 2 \\
\hline

           \multicolumn{1}{|c|}{BTCS306} &\multicolumn{2}{c|}{Data Structures} &  0 & 0 & 2 & 25 & 25 & 50 & 2 \\
\hline

\cline{3-7}
&\multicolumn{2}{c|}{TOTAL} & 15 & 10 & 9 & 275 & 375 & 650 & 16\\
\hline

\end{tabularx}
\clearpage
\center{\bf\color{black}{Punjab Technical University}}\linebreak
\bf\color{black}{}}
\center {\Large{{\color{black}\bf{BTCS 301 Computer Architecture}}}} 

\color{black}
\begin{itemize}
\item {\color{black}\large{\bf{Register Transfer and Microoperations : }}}\color{black}Register transfer language \& operations, arithmetic
microoperations, logic microoperations, shift microoperations, arithmetic logic shift unit. Design of a complete
basic computer and its working.
[ 5 ]\end{itemize}
\color{black}
\begin{itemize}
\item {\color{black}\large{\bf{ Basic Computer Organisation and Design : }}}\color{black}Instruction codes, Computer registers, Computer Instructions,
Timing and control, Instruction Cycle, Memory reference instructions, Input  Output and Interrupt, Design of
basic Computer, Design of Accumulator Logic.
[ 6 ]\end{itemize}
\color{black}
\begin{itemize}
\item {\color{black}\large{\bf{Design of Control Unit : }}}\color{black}Control memory, design of control unit � microprogrammed, hardwired, and their
comparative study.
[ 3 ]\end{itemize}
\color{black}
\begin{itemize}
\item {\color{black}\large{\bf{Central Processing Unit : }}}\color{black}General Register Organisation, Stack Organisation, Instruction formats, Addressing
Modes, Data transfer and manipulations, Program control, RISC and CISC architecture.
[ 6 ]\end{itemize}
\color{black}
\begin{itemize}
\item {\color{black}\large{\bf{Input-Output Organisation : }}}\color{black} Peripheral devices, IO Interface, asynchronous data transfer, modes of transfer,
priority interrupt, DMA, IO processor, serial communication.
[ 5 ]\end{itemize}
\color{black}
\begin{itemize}
\item {\color{black}\large{\bf{Memory Organisation : }}}\color{black}Memory hierarchy, main memory, auxiliary memory, associative memory, cache
memory, virtual memory, memory management hardware.
[ 6 ]\end{itemize}
\color{black}
\begin{itemize}
\item {\color{black}\large{\bf{Advanced concepts of Computer Architecture : }}}\color{black} Concept of pipeline, Arithmetic pipeline, Instruction , vector
processors and array processors. Introduction to parallel processing, Interprocessor communication \&
synchronization.
[ 5 ]\end{itemize}
\color{black}
\begin{itemize}
\item {\color{black}\large{\bf{Sets, relations and functions : }}}\color{black}Introduction, Combination of Sets, ordered pairs, proofs of general identities
of sets, relations, operations on relations, properties of relations and functions, Hashing Functions,
equivalence relations, compatibility relations, partial order relations.
[ 7 ]\end{itemize}
\color{black}
\begin{itemize}
\item {\color{black}\large{\bf{Combinatorial Mathematics : }}}\color{black}Basic counting principles Permutations and combinations Inclusion and
Exclusion Principle Recurrence relations, Generating Function, Application.
[ 7 ]\end{itemize}
\color{black}
\begin{itemize}
\item {\color{black}\large{\bf{Monoids and Groups : }}}\color{black}Groups Semigroups and monoids Cyclic semigraphs and submonoids, Subgroups
and Cosets. Congruence relations on semigroups. Morphisms. Normal subgroups. Dihedral groups.
[ 7 ]\end{itemize}
\color{black}
\begin{itemize}
\item {\color{black}\large{\bf{Graph Theory : }}}\color{black}Graph- Directed and undirected, Eulerian chains and cycles, Hamiltonian chains and cycles
Trees, Chromatic number Connectivity, Graph coloring, Plane and connected graphs, Isomorphism and
Homomorphism. Applications.
[ 3 ]\end{itemize}
\color{black}
\begin{itemize}
\item {\color{black}\large{\bf{Boolean Algebra : }}}\color{black}Boolean postulates and laws � De-Morgan�s Theorem, Principle of Duality, Boolean
expression � Boolean function, Minimization of Boolean expressions � Sum of Products (SOP), Product of
Sums (POS), Minterm, Maxterm, Canonical forms, Conversion between canonical forms, Karnaugh map
Minimization, Quine-McCluskey method - Don�t care conditions.
[ 5 ]\end{itemize}
\color{black}
\begin{itemize}
\item {\color{black}\large{\bf{Signal Conversions : }}}\color{black}Analog \& Digital signals. A,D and D,A conversion techniques (Weighted type, R-2R
Ladder type, Counter Type, Dual Slope type, Successive Approximation type).
[ 5 ]\end{itemize}
\color{black}
\begin{itemize}
\item {\color{black}\large{\bf{Stacks : }}}\color{black} Sequential and linked representations, operations on stacks, application of stacks such as parenthesis
checker, evaluation of postfix expressions, conversion from infix to postfix representation, implementing
recursive functions.
[ 4 ]\end{itemize}
\begin{flushleft}
{\color{black}\bf{Suggested Readings/ Books}}
\end{flushleft}
	
\begin{itemize}
\item  \color{black} M. Moris Mano, \textbf{Computer System Architecture,} P. Pal Choudhri.
\end{itemize}
	
\begin{itemize}
\item  \color{black} William Stallings, \textbf{Computer Organisation and Architecture,} P. Pal Choudhri.
\end{itemize}
	
\begin{itemize}
\item  \color{black} David A Patterson, \textbf{Computer Architecture,} P. Pal Choudhri.
\end{itemize}
	
\begin{itemize}
\item  \color{black} P. Pal Choudhri, \textbf{Computer Organisation and Design,} P. Pal Choudhri.
\end{itemize}
	
\begin{itemize}
\item  \color{black} Kai Hawang, \textbf{Advanced Computer Architecture,} A \& C BLACK.
\end{itemize}
\color{black}\line(1,0){550}
\clearpage
\clearpage
\center {\Large{{\color{black}\bf{BTCS302 Discrete Structures}}}} 

\color{black}
\begin{itemize}
\item {\color{black}\large{\bf{Rings and Boolean algebra : }}}\color{black}Rings, Subrings, morphism of rings ideals and quotient rings. Euclidean
domains Integral domains and fields Boolean Algebra direct product morphisms Boolean sub-algebra
Boolean Rings Application of Boolean algebra (Logic Implications, Logic Gates, Karnaugh-map)
[ 8 ]\end{itemize}
\begin{flushleft}
{\color{black}\bf{Suggested Readings/ Books}}
\end{flushleft}
	
\begin{itemize}
\item  \color{black} Lipschutz, \textbf{Discrete Mathematics (Schaum series),} fknkfnks.
\end{itemize}
\color{black}\line(1,0){550}
\clearpage
\clearpage
\center {\Large{{\color{black}\bf{BTCS303 Digital Circuits \& Logic Design}}}} 

\color{black}
\begin{itemize}
\item {\color{black}\large{\bf{Number Systems : }}}\color{black}Binary, Octal, Decimal, Hexadecimal. Number base conversions, 1s, 2s, rths
complements, signed Binary numbers. Binary Arithmetic, Binary codes: Weighted BCD, Gray code, Excess 3
code, ASCII � conversion from one code to another.
[ 5 ]\end{itemize}
\color{black}
\begin{itemize}
\item {\color{black}\large{\bf{Logic GATES : }}}\color{black}AND, OR, NOT, NAND, NOR, Exclusive-OR and Exclusive-NOR. Implementations of
Logic Functions using gates, NAND-NOR implementations. Study of logic families like RTL, DTL, DCTL,
TTL, MOS, CMOS, ECL and their characteristics.
[ 5 ]\end{itemize}
\color{black}
\begin{itemize}
\item {\color{black}\large{\bf{Combinational Circuits : }}}\color{black}Design procedure � Adders, Subtractors, Serial adder,Subtractor, Parallel adder,
Subtractor Carry look ahead adder, BCD adder, Magnitude Comparator, Multiplexer,Demultiplexer,
encoder,decoder, parity checker, code converters. Implementation of combinational logic using MUX.
[ 6 ]\end{itemize}
\color{black}
\begin{itemize}
\item {\color{black}\large{\bf{Sequential Circuits : }}}\color{black}Flip flops SR, JK, T, D and Master slave, Excitation table, Edge triggering, Level
Triggering, Realization of one flip flop using other flip flops. Asynchronous,Ripple counters, Synchronous
counters, Modulo-n counter, Ring Counters. Classification of sequential circuits-Moore and Mealy, Design of
Synchronous counters: state diagram, Circuit implementation. Shift registers.
[ 4 ]\end{itemize}
\color{black}
\begin{itemize}
\item {\color{black}\large{\bf{Memory Devices : }}}\color{black}Classification of memories, RAM organization, Write operation, Read operation, Memory
cycle. Static RAM Cell-Bipolar, RAM cell, MOSFET RAM cell, Dynamic RAM cell. ROM organization,
PROM, EPROM, EEPROM, Field Programmable Gate Arrays (FPGA).
[ 4 ]\end{itemize}
\begin{flushleft}
{\color{black}\bf{Suggested Readings/ Books}}
\end{flushleft}
\color{black}\line(1,0){550}
\clearpage
\center{\bf{Pariticals}}
\linebreak
\center{\bf{Digital Circuits \& Logic Design}}
\color{black}
\begin{itemize}
\item Study of Logic Gates: Truth-table verification of OR, AND, NOT, XOR, NAND and NOR gates; Realization of OR, AND, NOT and XOR functions using universal gates.\end{itemize}

\color{black}
\begin{itemize}
\item Half Adder / Full Adder: Realization using basic and XOR gates.\end{itemize}

\color{black}
\begin{itemize}
\item Half Subtractor / Full Subtractor: Realization using NAND gates.\end{itemize}

\color{black}
\begin{itemize}
\item 4-Bit Binary-to-Gray \& Gray-to-Binary Code Converter: Realization using XOR gates.\end{itemize}

\color{black}
\begin{itemize}
\item 4-Bit and 8-Bit Comparator: Implementation using IC7485 magnitude comparator chips.\end{itemize}

\color{black}
\begin{itemize}
\item Multiplexer: Truth-table verification and realization of Half adder and Full adder using IC74153 chip.\end{itemize}

\color{black}
\begin{itemize}
\item Demultiplexer: Truth-table verification and realization of Half subtractor and Full subtractor using IC74139 chip.\end{itemize}

\color{black}
\begin{itemize}
\item Flip Flops: Truth-table verification of JK Master Slave FF, T-type and D-type FF using IC7476 chip.\end{itemize}

\color{black}
\begin{itemize}
\item Asynchronous Counter: Realization of 4-bit up counter and Mod-N counter using IC7490 \& IC7493 chip.\end{itemize}

\color{black}
\begin{itemize}
\item Synchronous Counter: Realization of 4-bit up/down counter and Mod-N counter using IC74192 \& IC74193 chip.\end{itemize}

\color{black}
\begin{itemize}
\item Shift Register: Study of shift right, SIPO, SISO, PIPO, PISO \& Shift left operations using IC7495 chip.\end{itemize}

\color{black}
\begin{itemize}
\item DAC Operation: Study of 8-bit DAC (IC 08/0800 chip), obtain staircase waveform using IC7493 chip.\end{itemize}

\color{black}
\begin{itemize}
\item ADC Operations: Study of 8-bit ADC.\end{itemize}

\clearpage
\center {\Large{{\color{black}\bf{BTCS 304 Data Structures}}}} 

\color{black}
\begin{itemize}
\item {\color{black}\large{\bf{Dynamic Memory Management : }}}\color{black}Understanding pointers, usage of pointers, arithmetic on pointers, memory
allocation, memory management functions and operators, debugging pointers - dangling pointers, memory
leaks, etc.
[ 2 ]\end{itemize}
\color{black}
\begin{itemize}
\item {\color{black}\large{\bf{Introduction : }}}\color{black}Concept of data type, definition and brief description of various data structures, data structures
versus data types, operations on data structures, algorithm complexity, Big O notation.
[ 2 ]\end{itemize}
\color{black}
\begin{itemize}
\item {\color{black}\large{\bf{Arrays : }}}\color{black} Linear and multi-dimensional arrays and their representation, operations on arrays, sparse matrices
and their storage.
[ 3 ]\end{itemize}
\color{black}
\begin{itemize}
\item {\color{black}\large{\bf{Linked List : }}}\color{black} Linear linked list, operations on linear linked list, doubly linked list, operations on doubly linked
list, application of linked lists.
[ 4 ]\end{itemize}
\color{black}
\begin{itemize}
\item {\color{black}\large{\bf{Queues : }}}\color{black}equential representation of queue, linear queue, circular queue, operations on linear and circular
queue, linked representation of a queue and operations on it, deque, priority queue, applications of queues.
[ 4 ]\end{itemize}
\color{black}
\begin{itemize}
\item {\color{black}\large{\bf{Trees : }}}\color{black}Basic terminology, sequential and linked representations of trees, traversing a binary tree using
recursive and non-recursive procedures, inserting a node, deleting a node, brief introduction to threaded binary
trees, AVL trees and B-trees.
[ 4 ]\end{itemize}
\color{black}
\begin{itemize}
\item {\color{black}\large{\bf{Heaps : }}}\color{black}Representing a heap in memory, operations on heaps, application of heap in implementing priority
queue and heap sort algorithm.
[ 2 ]\end{itemize}
\color{black}
\begin{itemize}
\item {\color{black}\large{\bf{Graphs : }}}\color{black}Basic terminology, representation of graphs (adjacency matrix, adjacency list), traversal of a graph
(breadth-first search and depth-first search), and applications of graphs.
[ 3 ]\end{itemize}
\color{black}
\begin{itemize}
\item {\color{black}\large{\bf{Hashing \& Hash Tables : }}}\color{black}Comparing direct address tables with hash tables, hash functions, concept of
collision and its resolution using open addressing and separate chaining, double hashing, rehashing.
[ 3 ]\end{itemize}
\color{black}
\begin{itemize}
\item {\color{black}\large{\bf{Searching \& Sorting : }}}\color{black}Searching an element using linear search and binary search techniques, Sorting arrays
using bubble sort, selection sort, insertion sort, quick sort, merge sort, heap sort, shell sort and radix sort,
complexities of searching \& sorting algorithms.
[ 5 ]\end{itemize}
\begin{flushleft}
{\color{black}\bf{Suggested Readings/ Books}}
\end{flushleft}
	
\begin{itemize}
\item  \color{black} Sartaj Sahni, \textbf{Dynamic Memory Management,} hitesh.
\end{itemize}
	
\begin{itemize}
\item  \color{black} A. A. V. I. M., \textbf{Data Structures using C and C++,} A \& C BLACK.
\end{itemize}
\color{black}\line(1,0){550}
\clearpage
\center{\bf{Pariticals}}
\linebreak
\center{\bf{Data Structures}}
\color{black}
\begin{itemize}
\item Write a menu driven program that implements following operations (using separate functions) on a linear array:\end{itemize}

\color{black}
\begin{itemize}
\item Write a menu driven program that maintains a linear linked list whose elements are stored in on ascending order and implements the following operations (using separate functions)\end{itemize}

\color{black}
\begin{itemize}
\item Write a program to demonstrate the use of stack (implemented using linear array) in converting arithmetic expression from infix notation to postfix notation.\end{itemize}

\color{black}
\begin{itemize}
\item Program to demonstrate the use of stack (implemented using linear linked lists) in evaluating arithmetic expression in postfix notation.\end{itemize}

\color{black}
\begin{itemize}
\item Program to demonstration the implementation of various operations on a linear queue represented using a linear array.\end{itemize}

\color{black}
\begin{itemize}
\item Program to demonstration the implementation of various operations on a circular queue represented\end{itemize}

\color{black}
\begin{itemize}
\item Program to demonstration the implementation of various operations on a queue represented using a linear linked list (linked queue).\end{itemize}

\color{black}
\begin{itemize}
\item Program to illustrate the implementation of different operations on a binary search tree.\end{itemize}

\color{black}
\begin{itemize}
\item Program to illustrate the traversal of graph using breadth-first search.\end{itemize}

\color{black}
\begin{itemize}
\item Program to illustrate the traversal of graph using depth-first search.\end{itemize}

\color{black}
\begin{itemize}
\item Program to sort an array of integers in ascending order using bubble sort.\end{itemize}

\color{black}
\begin{itemize}
\item Program to sort an array of integers in ascending order using selection sort.\end{itemize}

\color{black}
\begin{itemize}
\item Program to sort an array of integers in ascending order using insertion sort.\end{itemize}

\color{black}
\begin{itemize}
\item Program to sort an array of integers in ascending order using radix sort.\end{itemize}

\color{black}
\begin{itemize}
\item Program to sort an array of integers in ascending order using merge sort.\end{itemize}

\color{black}
\begin{itemize}
\item Program to sort an array of integers in ascending order using quick sort.\end{itemize}

\color{black}
\begin{itemize}
\item Program to sort an array of integers in ascending order using heap sort.\end{itemize}

\color{black}
\begin{itemize}
\item Program to sort an array of integers in ascending order using shell sort.\end{itemize}

\color{black}
\begin{itemize}
\item Program to demonstrate the use of linear search to search a given element in an array.\end{itemize}

\color{black}
\begin{itemize}
\item Program to demonstrate the use of binary search to search a given element in a sorted array in ascending order.\end{itemize}

\clearpage
\center {\Large{{\color{black}\bf{BTCS 305 Object Oriented Programming Using C++}}}} 

\color{black}
\begin{itemize}
\item {\color{black}\large{\bf{Object-Oriented Programming Concepts : }}}\color{black}Introduction, comparision between procedural programming
paradigm and object-oriented proramming paradigm, basic concepts of object-oriented programming �
concepts of an object and a class, interface and implementation of a class, operations on objects, relationship
among objects, abstraction, encapsulation, data hiding, inheritance, overloading, polymorphism, messaging.
[ 2 ]\end{itemize}
\color{black}
\begin{itemize}
\item {\color{black}\large{\bf{Standard Input/Output : }}}\color{black}Concept of streams, hierarchy of console stream classes, input,output using
overloaded operators  and  and memberv functions of i,o stream classes, formatting output, formatting
using ios class functions and flags, formatting using manipulators.
[ 3 ]\end{itemize}
\color{black}
\begin{itemize}
\item {\color{black}\large{\bf{Classes and Objects : }}}\color{black}Specifying a class, creating class objects, accessing class members, access specifiers,
static members, use of const keyword, friends of a class, empty classes, nested classes, local classes, abstract
classes, container classes, bit fields and classes.
[ 4 ]\end{itemize}
\color{black}
\begin{itemize}
\item {\color{black}\large{\bf{Pointers and Dynamic Memory Management : }}}\color{black}Declaring and initializing pointers, accessing data through
pointers, pointer arithmetic, memory allocation (static and dynamic), dynamic memory management using new
and delete operators, pointer to an object, this pointer, pointer related problems - dangling,wild pointers, null
pointer assignment, memory leak and allocation failures.
[ 5 ]\end{itemize}
\color{black}
\begin{itemize}
\item {\color{black}\large{\bf{Constructors and Destructors : }}}\color{black}Need for constructors and destructors, copy constructor, dynamic
constructors, explicit constructors, destructors, constructors and destructors with static members, initializer
lists.
[ 2 ]\end{itemize}
\color{black}
\begin{itemize}
\item {\color{black}\large{\bf{Operator Overloading and Type Conversion : }}}\color{black}Overloading operators, rules for overloading operators,
overloading of various operators, type conversion - basic type to class type, class type to basic type, class type
to another class type.
[ 4 ]\end{itemize}
\color{black}
\begin{itemize}
\item {\color{black}\large{\bf{Inheritance : }}}\color{black}Introduction, defining derived classes, forms of inheritance, ambiguity in multiple and multipath
inheritance, virtual base class, object slicing, overriding member functions, object composition and delegation,
order of execution of constructors and destructors.
[ 5 ]\end{itemize}
\color{black}
\begin{itemize}
\item {\color{black}\large{\bf{Virtual functions \& Polymorphism : }}}\color{black}Concept of binding - early binding and late binding, virtual functions,
pure virtual functions, abstract clasess, virtual destructors
[ 3 ]\end{itemize}
\color{black}
\begin{itemize}
\item {\color{black}\large{\bf{Exception Handling : }}}\color{black}Review of traditional error handling, basics of exception handling, exception handling
mechanism, throwing mechanism, catching mechanism, rethrowing an exception, specifying exceptions.
[ 2 ]\end{itemize}
\color{black}
\begin{itemize}
\item {\color{black}\large{\bf{Templates and Generic Programming : }}}\color{black}Template concepts, Function templates, class templates, illustrative
examples
[ 3 ]\end{itemize}
\color{black}
\begin{itemize}
\item {\color{black}\large{\bf{Files : }}}\color{black}File streams, hierarchy of file stream classes, error handling during file operations, reading,writing of
files, accessing records randomly, updating files
[ 3 ]\end{itemize}
\begin{flushleft}
{\color{black}\bf{Suggested Readings/ Books}}
\end{flushleft}
	
\begin{itemize}
\item  \color{black} Lafore R, \textbf{Object Oriented Programming in C++,} William Stallings.
\end{itemize}
	
\begin{itemize}
\item  \color{black} E. Balagurusamy, \textbf{Object Oriented Programming with C++,} Bjarne Stroustrup.
\end{itemize}
	
\begin{itemize}
\item  \color{black} R. S. Salaria, \textbf{Mastering Object-Oriented Programming with C++,} Sartaj Sahni.
\end{itemize}
\color{black}\line(1,0){550}
\clearpage
\center{\bf{Pariticals}}
\linebreak
\center{\bf{Object Oriented Programming Using C++}}
\color{black}
\begin{itemize}
\item Write a program that uses a class where the member functions are defined inside a class\end{itemize}

\color{black}
\begin{itemize}
\item Write a program that uses a class where the member functions are defined outside a class.\end{itemize}

\color{black}
\begin{itemize}
\item Write a program to demonstrate the use of static data members.\end{itemize}

\color{black}
\begin{itemize}
\item Write a program to demonstrate the use of const data members\end{itemize}

\color{black}
\begin{itemize}
\item Write a program to demonstrate the use of zero argument and parameterized constructors.\end{itemize}

\color{black}
\begin{itemize}
\item Write a program to demonstrate the use of dynamic constructor.\end{itemize}

\color{black}
\begin{itemize}
\item Write a program to demonstrate the use of explicit constructor.\end{itemize}

\color{black}
\begin{itemize}
\item Write a program to demonstrate the use of initializer list.\end{itemize}

\color{black}
\begin{itemize}
\item Write a program to demonstrate the overloading of increment and decrement operators.\end{itemize}

\color{black}
\begin{itemize}
\item Write a program to demonstrate the overloading of binary arithmetic operators.\end{itemize}

\color{black}
\begin{itemize}
\item Write a program to demonstrate the overloading of memory management operators.\end{itemize}

\color{black}
\begin{itemize}
\item Write a program to demonstrate the typecasting of basic type to class type.\end{itemize}

\color{black}
\begin{itemize}
\item Write a program to demonstrate the typecasting of class type to basic type.\end{itemize}

\color{black}
\begin{itemize}
\item Write a program to demonstrate the multilevel inheritance.\end{itemize}

\clearpage

\center{\Large{{\huge{\bf{Scheme and Syllabus of}}}}}
\center{{\huge{\bf{ B.Tech}}}
\center{{\huge{ \bf{Computer Science Engineering}}}}
\center{{\Large{ \color{black}\bf{4} Semester effective for Batch }}}
\center{{\Huge{ \color{black}\emph{\linebreak\linebreak\linebreak\linebreak 4   SEMESTER}}}}
\linebreak\linebreak\linebreak
\begin{tabularx}{\linewidth}{|*{15}{X|}}
\hline
\multirow{2}{*}{Code} &\multicolumn{2}{|c|}{Course Details} &\multicolumn{3}{c|}{Load Allocation}&\multicolumn{2}{c|}{Mark Distribution} &\multicolumn{1}{|c|}{Total Marks}&\multicolumn{1}{|c|}{Credit} \\
\cline{4-8}
           & \multicolumn{2}{|c|}{} &  L & T & P & Internal & External & & \\

\hline
\cline{3-7}
           \multicolumn{1}{|c|}{BST-307} &\multicolumn{2}{c|}{nklllllllllll} &  3 & 2 & 1 & 40 & 60 & 100 & 2 \\
\hline

\cline{3-7}
&\multicolumn{2}{c|}{TOTAL} & 3 & 2 & 1 & 40 & 60 & 100 & 2\\
\hline

\end{tabularx}
\clearpage
\center{\bf\color{black}{Punjab Technical University}}\linebreak
\bf\color{black}{}}
\center {\Large{{\color{black}\bf{BST-307 nklllllllllll}}}} 

\begin{flushleft}
{\color{black}\bf{Suggested Readings/ Books}}
\end{flushleft}
\color{black}\line(1,0){550}
\clearpage
\clearpage

\end{document}
